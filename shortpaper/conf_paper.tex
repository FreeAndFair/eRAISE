% words to eliminate/reduce: therefore


%% conf_paper.tex

\documentclass[conference]{IEEEtran}
% Add the compsoc option for Computer Society conferences.
%
% If IEEEtran.cls has not been installed into the LaTeX system files,
% manually specify the path to it like:
% \documentclass[conference]{../sty/IEEEtran}


% *** GRAPHICS RELATED PACKAGES ***
%
\ifCLASSINFOpdf
   \usepackage[pdftex]{graphicx}
  % declare the path(s) where your graphic files are
   \graphicspath{{../pdf/}{../jpeg/}}
  % and their extensions so you won't have to specify these with
  % every instance of \includegraphics
   \DeclareGraphicsExtensions{.pdf,.jpeg,.png}
\else
  % or other class option (dvipsone, dvipdf, if not using dvips). graphicx
  % will default to the driver specified in the system graphics.cfg if no
  % driver is specified.
   \usepackage[dvips]{graphicx}
  % declare the path(s) where your graphic files are
   \graphicspath{{../eps/}}
  % and their extensions so you won't have to specify these with
  % every instance of \includegraphics
   \DeclareGraphicsExtensions{.eps}
\fi

%=========================================================
% Latex Packages
%=========================================================
\usepackage{listings}
\usepackage{xspace}
\usepackage[colorlinks=true,linkcolor=blue,citecolor=blue,urlcolor=blue]{hyperref}
\usepackage[colorinlistoftodos,textwidth=2.5cm,shadow]{todonotes}

%=========================================================
% Other
%=========================================================

\usepackage{color}
\usepackage{listings}


%\newcommand{\comment}[1]{\textcolor{red}{#1}}
\definecolor{keywordcolor}{rgb}{0.5,0,0.33}
\definecolor{identifiercolor}{rgb}{0,0,0.75}
%\definecolor{commentcolor}{rgb}{0.25,0.5,0.37}
\definecolor{commentcolor}{rgb}{0.3,0.3,0.3} 

\lstdefinelanguage{bon} {
  morekeywords={class_chart,indexing,explanation,part,query,command,constraint,
  end,deferred,effective,persistent,require,ensure,invariant,feature,class,
  static_diagram,component,old,not,inherit,delta,for_all,such_that,it_holds,Current,Lockset,when,monitors_for,max,concurrency,concurrent,guarded,locks,special,failure,sequential,atomic},
  morekeywords={[2]LOCK,SEMAPHORE,BOOLEAN,INTEGER,REAL,SEQUENCE,MAIN,EXCEPTION,NO_SEATS,BARBER_SHOP,CUSTOMER,BARBER},
  morekeywords={[3]},
  morecomment=[l]{--}, morestring=[b]", morestring=[d]'
  }[keywords,comments,strings]

\lstdefinestyle{bon}{language={bon},showstringspaces={false},
  basicstyle={\scriptsize\ttfamily\mdseries},
  keywordstyle={\color{keywordcolor}},
  keywordstyle={[2]\color{black}\bfseries},
  keywordstyle={[3]\color{black}\bfseries},
  identifierstyle={\color{identifiercolor}},
  commentstyle={\color{commentcolor}},
  frame=lines}

\lstdefinestyle{boninline}{language={bon},showstringspaces={false},
  basicstyle={\ttfamily\mdseries},
  keywordstyle={\color{black}},
  keywordstyle={[2]\color{black}\bfseries},
  keywordstyle={[3]\color{black}\bfseries},
  identifierstyle={\color{black}\mdseries},
  commentstyle={\color{black}},
  frame=lines,
  columns=fullflexible,
  breaklines=true}
\lstset{style=bon, columns=fullflexible, keepspaces=true, 
        frame=topline, captionpos=b}

%=========================================================
% Generic Macros
%=========================================================

\newcommand{\lstbon}[1]{\lstinline[style=boninline]{#1}} 
\newcommand{\lstjava}[1]{\lstinline[style=jml2inline]{#1}} 

\newcommand{\eg}{e.g.,\xspace}
\newcommand{\ie}{i.e.,\xspace}
\newcommand{\etc}{etc.\xspace}
\newcommand{\vs}{vs.\xspace}
\newcommand{\etal}{et.~al.\xspace}

% Todos
\newcommand{\note}[1]{\todo[inline,color=red!40]{#1}}

% correct bad hyphenation here
\hyphenation{op-tical net-works semi-conduc-tor}

% =========================================================

\begin{document}
%
% paper title
% can use linebreaks \\ within to get better formatting as desired
\title{A Rigorous Methodology for Analyzing and Designing Plug-ins}


% author names and affiliations
% use a multiple column layout for up to three different
% affiliations
\author{
\IEEEauthorblockN{Marieta V. Fasie}
\IEEEauthorblockA{DTU Compute\\Technical University of Denmark\\
DK-2800 Lyngby, Denmark\\
Email: marietafasie@gmail.com}

\and

\IEEEauthorblockN{Anne E. Haxthausen}
\IEEEauthorblockA{DTU Compute\\Technical University of Denmark\\
DK-2800 Lyngby, Denmark\\
Email: ah@imm.dtu.dk}

\and

\IEEEauthorblockN{Joseph R. Kiniry}
\IEEEauthorblockA{DTU Compute\\Technical University of Denmark\\
DK-2800 Lyngby, Denmark\\
Email: jkin@imm.dtu.dk}}

% use for special paper notices
%\IEEEspecialpapernotice{(Invited Paper)}


% make the title area
\maketitle
\thispagestyle{plain}
\pagestyle{plain}

% =========================================================
% Abstract
% =========================================================

\begin{abstract}
%\boldmath
  Today, GUI plug-ins development is typically done in a very ad-hoc
  way, where developers dive directly into implementation.  Without
  any prior analysis and design, plug-ins are often flaky, unreliable,
  difficult to maintain and extend with new functionality, and have
  inconsistent user interfaces.  This paper addresses these problems
  by describing a rigorous methodology for analyzing and designing
  plug-ins.  The methodology is grounded in the Extended Business
  Object Notation (EBON) and covers informal analysis and design of
  features, GUI, actions, and scenarios, formal architecture design,
  including behavioral semantics, and validation.  The methodology is
  illustrated via a case study whose focus is an Eclipse environment
  for the RAISE formal method's tool suite.
\end{abstract}

% no keywords

% For peer review papers, you can put extra information on the cover
% page as needed:
% \ifCLASSOPTIONpeerreview
% \begin{center} \bfseries EDICS Category: 3-BBND \end{center}
% \fi
%
% For peerreview papers, this IEEEtran command inserts a page break and
% creates the second title. It will be ignored for other modes.
\IEEEpeerreviewmaketitle


% =========================================================
% Introduction
% =========================================================
\section{Introduction}
\label{sec:introduction}

Plug-ins, especially in the realm of plug-ins that wrap existing
research command-line tools, are notoriously badly designed.
Academics simply do not have the resources and expertise to execute on
the design and implementation of a quality plug-in.  Partly this is due
to the fact that there are few examples of best practices in the area,
and partly it is because plug-in development is viewed as the dirtiest
of the dirty-but-necessary jobs of ``selling'' systems technology.

Typically, a researcher has developed a novel tool for Java
programming, lets call it the \texttt{CommandLineWidget}.  They want
others to use this tool, but few people these days want to mess about
with downloading and building source code and, sadly, the barrier to
entry for command-line tools is not insignificant these days.
Instead, the researcher wants to ``sell'' their tool by wrapping it in
the \texttt{CommandLineFeature} for Eclipse, since Eclipse has the
mind-share of most Java developers.  But the researcher does not know
how to think about the UI design of an Eclipse plug-in, design and
program the plug-in, nor is she really interested in learning how to
do these things.

Eclipse plug-in development is a tricky world.  Concepts like
features, plug-ins, extension points, windows, views, \etc abound.
Enormous, poorly documented APIs are prolific in the Eclipse
ecosystem.  To implement even the most basic of features sometimes
takes hours of digging to find the right three lines of code, and then
those lines must change when a new major version of Eclipse comes out.
This is a frustrating world for researchers who want to package their
demonstrable, useful tools for the Eclipse IDE.

This work is an attempt to help resolve these issues.  First, we
provide a \emph{rigorous step-wise methodology through which one can
  do the analysis, architecture design, and user interface (UI) design
  of a plug-in for an arbitrary integrated development environment
  (IDE)}.  Second, we provide \emph{template example plug-ins} that
can be reused by a programmer with a good understanding of Java, but a
poor understanding of Eclipse plug-in development, to develop a new
Eclipse plug-in or feature.

The methodology used is based upon the Business Object Notation (BON),
an analysis and design methodology promoted by Walden and Nerson in
the mid-90s within the Eiffel community~\cite{WaldenNerson95}.
Ostroff, Paige, and Kiniry formalized parts of the BON language and
reasoned about BON
specifications~\cite{LancaricOstroffPaige02,EBON01,PaigeEtal02,PaigeOstroff01b}.
Fairmichael, Kiniry, and Darulova developed the BONc and Beetz tools
for reasoning about BON specifications and their refinement to
JML-annotated Java.\footnote{See \url{http://tinyurl.com/brgcrzc} for
  more information.}  Finally, Kiniry and Fairmichael have extended
BON in a variety of ways to produce Extended BON (EBON), which permits
one to add new domain-specific syntax and semantics to the core BON
language~\cite{Kiniry02-PhDThesis}.

For the reader who has never heard of EBON, think of it as the subset
of UML that might actually have a clear, unambiguous semantics.
EBON's core features are that it is \emph{seamless}, insofar as you
use the same specification language for everything from domain
analysis to formal architecture specification and its behavior,
\emph{reversible} insofar as code generation and reverse engineering
to and from code to EBON is straightforward and tool-supported, and
\emph{contracted} as formal abstract state-based contracts
(invariants, pre, and postconditions) are the fundamental notion used
to specify system behavior.  EBON has both a textual and a graphical
syntax, a formal semantics expressed in higher-order logic, a formal
semantics of refinement to and from OO software, and tool support for
reasoning about specifications, expressing specifications textually or
graphically, generating code from models and models from code, and
reasoning about refinement to code.

The methodology is illustrated on a case study that develops an
Eclipse environment for the RAISE formal method~\cite{RMG95} and
specification language (RSL)~\cite{RLG92}. The project is called
\emph{eRAISE} and it is currently under development at DTU.


The \emph{rsltc} tool suite~\cite{rsltcUserGuide,RAISETools2003} consists of a type checker and some
extensions to it supporting activities such as pretty printing,
extraction of module dependencies, translation to other languages,
generation of proof obligations, formal verification, and generation
and execution of test cases.  \emph{rsltc} has a command-line
interface that exposes different capabilities selected via switches,
but is also used from Emacs using menus and key-binding.
However, although it is easy to use for the user comfortable with
command-line tools or Emacs, we expect that the creation of a modern
Eclipse-based development environment for \emph{rsltc} would broaden
its appeal to mainstream software engineers and better enable its use
for university-level pedagogy.

% Related work
%
\section{Related work}
\label{sec:related-work}

There is little published work that focuses on methodologies specific
to plug-in development.  E.g., Lamprecht et al. reflect over some
simplicity principles elicited by many years' experience in plug-in
development~\cite{6229816}, but do not provide a methodology.

We speculate that there is not much published work because plug-in
development was not the focus of scientists until recently.
Moreover, it is a fair question whether or not plug-in development is
any different from normal systems development where a GUI is involved.
We believe that plug-in development is different from normal GUI
development as plug-ins must integrate into the larger framework
of the IDE, deal with non-GUI events, and work in arbitrary
compositions.

% The EBON method has been applied to not only OO software systems, but
% websites and business processes, thus in some sense it should be
% unsurprising that it can be used for plug-in development.  We also
% borrow the UI sketching principles of the 

% =========================================================
% Analysis and design method
% =========================================================
\section{Analysis and Design Method}
\label{sec:analys-design-meth}

The EBON methodology as applied to Eclipse plug-in development has six
stages, each described in a separate paragraph and presented in the
order in which they are applied. These six steps are: \emph{domain
  modeling}, \emph{user interface}, \emph{events}, \emph{components},
\emph{components communication} and \emph{code generation}.

% Domain Modeling
%
{\bf Step 1: Domain Modeling.} 
In the first step the most
important entities and high level classifiers related to the system
domain are identified, explained and documented. The identified
notions are documented as classes, which can be
grouped under clusters and all these make up a unique system.
\autoref{example:system_chart} illustrates a caption of the eRAISE
System specified in EBON notation

\lstset{style={bon}}
\lstinputlisting[style=bon, float=tp,label={example:system_chart},
caption={System chart describing the eRAISE system.},
captionpos=b]{system_chart_example.bon}

The domain model also describes how concepts behave and how their
behavior is constrained. In \autoref{example:system_chart}, \emph{class\_chart
  Console} offers the service of displaying informative messages or
error messages with the \emph{constraint}
that it must be cleared before displaying a new
message.


% User Interface
%
{\bf Step 2: User Interface.} 
This step determines all the things a
user can do from the plug-in's UI by creating a mock-up user interface
for each user action that is relevant
and important for the plug-in. While the user interface is being
drawn, product requirements are
documented using EBON \emph{scenario\_charts}.  
\autoref{example:scenario_chart} presents the requirement for a 
sub-menu item of RSL menu.

\lstset{style={bon}}
\lstinputlisting[style=bon, float=tp,label={example:scenario_chart},
caption={Scenario chart for the RSL menu item.},
captionpos=b]{scenario_example.bon}

% Events
%
{\bf Step 3: Events.}
In this stage, the system is seen as a black
box. The focus is on the external actions that make the system react
and on the system's outgoing responses. An incoming external event is
any action that determines the system to
change its state while an outgoing internal event is the
response the system sends as a reaction to an incoming external event.
\autoref{example:event_chart_in} illustrates an incoming external
event named
\emph{TYPECHECKALL}, triggered by a user action. The system response
to
this action is captured in an EBON \emph{outgoing event} in
\autoref{example:event_chart_out} .

\lstset{style={bon}}
\lstinputlisting[style=bon, float=tp,label={example:event_chart_in},
caption={Incoming event chart for typechecking features.},
captionpos=b]{external_event_example.bon}

\lstset{style={bon}}
\lstinputlisting[style=bon, float=tp,label={example:event_chart_out},
caption={Outgoing event chart for typechecking features.},
captionpos=b]{internal_event_example.bon}

% Components
%
{\bf Step 4: Components.}
This stage looks \emph{inside} the system at the components that constitute
its architecture. 

% Components Communication
%
{\bf Step 5: Components Communication.}

% Code Generation
%
{\bf Step 6: Code Generation.}
In the last step a tool named Beetlz~\cite{Darulova09} is applied to 
automatically generate JML-annotated, Javadoc documented Java code 
from the EBON \emph{system\_chart} and
\emph{static\_diagrams} created in previous steps. 
Beetlz also performs refinement analysis so that architecture drift is
automatically identified as the system evolves, either at the
model-level in EBON, or within the implementation in Java.


\bibliographystyle{IEEEtran}
\bibliography{abbrev,main-bib}

\end{document}
