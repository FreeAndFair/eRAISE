
%% conf_paper.tex

\documentclass[conference]{IEEEtran}
% Add the compsoc option for Computer Society conferences.
%
% If IEEEtran.cls has not been installed into the LaTeX system files,
% manually specify the path to it like:
% \documentclass[conference]{../sty/IEEEtran}


% *** GRAPHICS RELATED PACKAGES ***
%
\ifCLASSINFOpdf
  % \usepackage[pdftex]{graphicx}
  % declare the path(s) where your graphic files are
  % \graphicspath{{../pdf/}{../jpeg/}}
  % and their extensions so you won't have to specify these with
  % every instance of \includegraphics
  % \DeclareGraphicsExtensions{.pdf,.jpeg,.png}
\else
  % or other class option (dvipsone, dvipdf, if not using dvips). graphicx
  % will default to the driver specified in the system graphics.cfg if no
  % driver is specified.
  % \usepackage[dvips]{graphicx}
  % declare the path(s) where your graphic files are
  % \graphicspath{{../eps/}}
  % and their extensions so you won't have to specify these with
  % every instance of \includegraphics
  % \DeclareGraphicsExtensions{.eps}
\fi



% correct bad hyphenation here
\hyphenation{op-tical net-works semi-conduc-tor}


\begin{document}
%
% paper title
% can use linebreaks \\ within to get better formatting as desired
\title{A Method for Analyzing and Designing GUI Plug-ins}


% author names and affiliations
% use a multiple column layout for up to three different
% affiliations
\author{\IEEEauthorblockN{ Anne E. Haxthausen}
\IEEEauthorblockA{DTU Compute\\Technical University of Denmark\\
DK-2800 Lyngby, Denmark\\
Email: ah@imm.dtu.dk}
\and

\IEEEauthorblockN{Joseph Kiniry}
\IEEEauthorblockA{DTU Compute\\Technical University of Denmark\\
DK-2800 Lyngby, Denmark\\
Email: jkin@imm.dtu.dk}
\and

\IEEEauthorblockN{ Marieta V. Fasie}
\IEEEauthorblockA{DTU Compute\\Technical University of Denmark\\
DK-2800 Lyngby, Denmark\\
Email: marietafasie@gmail.com}}



% use for special paper notices
%\IEEEspecialpapernotice{(Invited Paper)}


% make the title area
\maketitle

%------------------
% Abstract
%------------------

\begin{abstract}
%\boldmath
Today, GUI plug-ins development is made in an ad-hoc way and typically starts directly with the implementation phase.
Without a prior analysis and design, the final plug-in is unreliable, difficult to maintain and extend with new functionalities. 
The current paper addresses these problems by describing a systematic method for analyzing and designing GUI plug-in systems. 
The method is based on the Business Object Notation approach and consists of a number of well-defined steps. 
Furthermore, the method is illustrated on a study case which develops an Eclipse environment for the RAISE tool set.
\end{abstract}

% no keywords

% For peer review papers, you can put extra information on the cover
% page as needed:
% \ifCLASSOPTIONpeerreview
% \begin{center} \bfseries EDICS Category: 3-BBND \end{center}
% \fi
%
% For peerreview papers, this IEEEtran command inserts a page break and
% creates the second title. It will be ignored for other modes.
\IEEEpeerreviewmaketitle


%------------------
% Introduction
%------------------
\section{Introduction}
What is the paper about.


% Background
%
\subsection{Background}
What problems do we run into when starting building an Eclipse plug-in.



% Related work
%
\subsection{Related work}
What solutions have other papers brought


%------------------
% Analysis and design method
%------------------
\section{Analysis and design method}
The steps used to design and analyze the plug-in.


% User interface
%
\subsection{User interface}
UI mock-ups. 


Requirements identification. Captured in BON \emph{scenario\_chart}.

% Events
%
\subsection{Events}
\emph{Incoming} events representing user actions and \emph{outgoing} events meant to inform the user.


% Components
%
\subsection{Components}
Major components captured in BON \emph{static\_diagram}s using \emph{cluster\_chart} and \emph{class}.


% Components communication
%
\subsection{Components communication}
Component interfaces added to the interface diagram using  \emph{feature}, \emph{require} and \emph{ensure}. This will later result in plug-in extensions and extension points.


Update scenarios with events.

% Code generation
%
\subsection{Code generation}
Beetlz generates the Java code from BON specification.

%------------------
% Conclusion
%------------------
\section{Conclusion}
In conclusion



% conference papers do not normally have an appendix


% use section* for acknowledgement
\section*{Acknowledgment}


The authors would like to thank...




% trigger a \newpage just before the given reference
% number - used to balance the columns on the last page
% adjust value as needed - may need to be readjusted if
% the document is modified later
%\IEEEtriggeratref{8}
% The "triggered" command can be changed if desired:
%\IEEEtriggercmd{\enlargethispage{-5in}}

% references section

% can use a bibliography generated by BibTeX as a .bbl file
% BibTeX documentation can be easily obtained at:
% http://www.ctan.org/tex-archive/biblio/bibtex/contrib/doc/
% The IEEEtran BibTeX style support page is at:
% http://www.michaelshell.org/tex/ieeetran/bibtex/
%\bibliographystyle{IEEEtran}
% argument is your BibTeX string definitions and bibliography database(s)
%\bibliography{IEEEabrv,../bib/paper}
%
% <OR> manually copy in the resultant .bbl file
% set second argument of \begin to the number of references
% (used to reserve space for the reference number labels box)
%
\begin{thebibliography}{1}

\bibitem{IEEEhowto:kopka}
H.~Kopka and P.~W. Daly, \emph{A Guide to \LaTeX}, 3rd~ed.\hskip 1em plus
 0.5em minus 0.4em\relax Harlow, England: Addison-Wesley, 1999.

\end{thebibliography}

% that's all folks
\end{document}


